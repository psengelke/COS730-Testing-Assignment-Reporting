
\section{Overview of Test Results}
\label{sec:overviewResults}
The tests for this module have all failed. The main reasons will be described in this document. Unit tests have been compiled according to the specification. We utilized a mocking framework, Mockito, for the unit tests on the module. Two main services were required for this module. Namely, getAccreditationUnitsReport and  getProgressReport. Tests will be conducted to examine whether reports can be generated for the different scenarios specified in the section titled Test Identification in the Unit Test Plan. These scenarios include the filtering options for the AccreditationUnits service and these are: per entity(group or person), per publication type, per selection of lifecycle states and any combination of these. The progress report can only be filtered per entity(group or person), per publication type or both. Unit tests were written to evaluate these. The essential non-functional requirements of the module, identified in the Software Requirements Specification were are also evaluated and the results are described in this report. 
\section{Detailed Test Results} 
\label{sec:detailedResults}  
\subsection{Functional Requirements Test Results}
The tests we have created for this module are contained within the file ReportingTest.java which can be found in the following location:
\\  COS730-Testing-Assignment-Reporting/software/reporting/src/test/java/org/cos730/bugsrus/reporting.

\subsubsection{Accreditation Reporting By Entity (TC \ref{ti_ar_entity})}  
The following results were obtained from the tests conducted. The tests to produce the following results have failed and the reasons are stated below. 
\begin{enumerate}
	\item A report containing information regarding or related to the accreditation target of the specified researcher could not be generated due to the fact the functionality did not conform to the service contracts stipulated in the RDS. 
	\item A report containing information regarding or related to the accreditation target of the specified research group could not be generated due to the fact the functionality did not conform to the service contracts stipulated in the RDS.
	\item An empty report for the last two cases could also not be generated for lack of conformance, but also because Jasper Reports were not utilised. 
\end{enumerate}
\paragraph{Result:} Fail.

\subsubsection{Accreditation Reporting By Publication Type (TC \ref{ti_ar_pubtype})}
\begin{enumerate}
	\item A report of all publications of PublicationType: Journal, where the accreditation unit value, if published, cannot be generated because there is no functionality to support filtering by publication type furthermore there is no use of Jasper Reports.
	\item A report of all publications of PublicationType: Conference, where the accreditation unit value, if published, cannot be generated because of the same reasons.  
	\item A report of all publications of PublicationType: Book Chapter, where the accreditation unit value, cannot be generated as well because of the same reasons stated above.
\end{enumerate}
\paragraph{Result:} Fail.

\subsubsection{Accreditation Reporting By Life-cycle State (TC \ref{ti_ar_lcs})}
For each of the inputs above, a report of all publications in that state exclusively, where the accreditation unit value is included cannot be generated because a function to filter by LifeCycleState has not been included in the classes provided and again due to the lack of a reporting framework, a report cannot be generated.
\paragraph{Result:} Fail.

\subsubsection{Accreditation Reporting By Some Combination (TC \ref{ti_ar_com})}
Each combination must return a report that only shows information that is within the filters' constraints, since none of the accreditation filters were implemented this is not possible.
\paragraph{Result:} Fail.
 
\subsubsection{Progress Reporting By Entity (TC \ref{ti_sr_entity})} 
The functionality was not implemented and thus fails all input tests.
\paragraph{Result:} Fail.

\subsubsection{Progress Reporting By Publication Type (TC \ref{ti_sr_pubtype})}
For each PublicationType as input, the InProgress publications of that type could not be returned in the report due to the fact that there was no function created to service this request and to create a report based on filters.
\paragraph{Result:} Fail.

\subsubsection{Progress Reporting By Some Combination (TC \ref{ti_sr_com})}
A report filtered by both Entity and PublicationType could not be returned since no functionality was implemented with the required filters.
\paragraph{Result:} Fail.

\subsection{Non-functional Requirements Test Results}
\begin{enumerate} 
\item Flexibility - The code provided is not flexible because firstly Jasper reports was not used and thus there is no place where Jasper reports can be easily plugged in for use. Hardcoding has taken place in various places in the code such as the date for the report has been set to 1 Jan 2016 in the constructor of the Entity class and the Research Group progress has also been set to 80 percent in the Group class's constructor. This does not allow for flexibility because the values of such attributes would need to be changed in the code manually. In terms of persistence pluggability, no persistence has been used in the code thus this does not indicate whether or not there can be flexible insertion and modifiability of the persistence provider. Due to the fact that persistence has not been inserted, it can be deduced that the code is not flexible to allow for the use of persistence.
\item Maintainability - The technologies and frameworks used in the project i.e. Java EE, EJB and Maven are quite widely used in industry and there is a sufficient amount of documentation available on these to aid future developers who will work on systems that use them. However, the code at hand does not have in code comments or well documente functions stating the purpose, the parameters and return values and this hence contributes to the fact that the code is not readable and thus it is not maintainable. In addition, due to the fact that no flexibility and pluggability is used, this would make maintenance difficult because one would need go deep into the code to make changes and would need to change more than one part of the code. There are various classes contained in the module, but the purpose of these is not documented and upon further inspection these classes merely consist of getters and setters, which would further confuse the developer as to what function they perform. 
\item Reliability - The code provided is not reliable due to the fact that the purpose of each class is not stated, the module implementation does not conform to the specification and pre and post conditions and the classes' attributes have been hardcoded.
\item Deployability - The code was not deployed on a server but it would not be ideal to deploy it on a server as a report cannot be generated and hence the functionality is not complete or correct. 
\end{enumerate}
\section{Other Tests and Evaluations}
 
\begin{enumerate}    
\item Provide the Maven build with appropriate dependencies
\begin{figure}[h!]
	\includegraphics[scale=0.6]{./figures/mavenBuild1.PNG}
	\caption{The Maven Build}
\end{figure} 

\begin{figure}[h!]
	\includegraphics[scale=0.6]{./figures/mavenBuild2.PNG}
	\caption{The tests from the repo provided.}
\end{figure}
\item Use of Jasper Reports - Jasper Reports has not been used in the code and thus no reports are generated. There is no provision in the code for a framework such as Jasper Reports. 
\item Use of contracts and mock objects - The contracts specified in the Requirements Specification document have not been adhered to or implemented in the module. More specifically the two main services getAccreditationUnitsReport and getProgressReport and their functionality has not been implemented. In terms of mocking, no mocks have been used in the implementation and no mocking framework such as Mockito has been used in the unit tests provided either.
\item Deployable into application server -The code was not deployed on a server but it would not be ideal to deploy it on a server as a report cannot be generated and hence the functionality is not complete or correct.  
\end{enumerate}

\section{Conclusions and Recommendations}
\label{sec:conclusions}
There were various limitations identified in the testing procedure and these were the main reasons for the failure of the unit tests. Firstly the fact that the module was not implemented according to the contract and secondly the fact that no persistence was used. This is a problem because reports generated would need information from persistence to display. In addition, no reporting framework was used, such as Jasper Reports, which serves as a problem because no report could be generated in the assumed format. Mock objects were not used and thus no mocking framework was used to provide mocks to the unit tests. The tests provided by the group do not test the pre and post conditions and specifications. Hence, we implemented tests to test the services specified in the contracts. Due to the fact that these services were not implemented, the tests failed. Futhermore, the pass criteria specified in the section titled Pass/Fail Criteria inside the Test Plan was not met and hence the tests are considered failed. 

\newpage
\appendix
\section{Git Contributions}
\begin{figure}[h!]
	\includegraphics[scale=0.6]{./figures/gitGraph.PNG}
	\caption{Git Contributions}
\end{figure}

\section{Conclusions and Recommendations}
\label{sec:conclusions}
There were various limitations identified in the testing procedure and these were the main reasons for the failure of the unit tests. Firstly the fact that the module was not implemented according to the contract and secondly the fact that no persistence was used. This is a problem because reports generated would need information from persistence to display. In addition, no reporting framework was used, such as Jasper Reports, which serves as a problem because no report could be generated in the assumed format. Mock objects were not used and thus no mocking framework was used to provide mocks to the unit tests. The tests provided by the group do not test the pre and post conditions and specifications. Hence, we implemented tests to test the services specified in the contracts. Due to the fact that these services were not implemented, the tests failed. Futhermore, the pass criteria specified in the section titled Pass/Fail Criteria inside the Test Plan was not met and hence the tests are considered failed. We utilized a mocking framework, Mockito to create mocks for the unit tests. We also mocked Jasper Reports in the unit tests for accuracy purposes. Request and responses were created for two services, Accreditation Points Report and the Progress Report. The tests can be located in the folder specified in the overview section above.

