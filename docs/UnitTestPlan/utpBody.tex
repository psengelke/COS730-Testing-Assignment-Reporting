%%%%%%%%%%%%%%%%%%%%%%%%%%%%% GUIDELINES %%%%%%%%%%%%%%%%%%%%%%%%%%%%%%%%%
%
%	Make the plan concise. Avoid redundancy and superfluousness.
% 	If you think you do not need a section that has been mentioned
% 	in the template, go ahead and delete that section in your
% 	test plan.
%
% 	Be specific. For example, when you specify an operating system
% 	as a property of a test environment, mention the OS Edition/Version
% 	as well, not just the OS Name.
%
%	Make use of lists and tables wherever possible. Avoid lengthy
% 	paragraphs.
%
%	Have the test plan reviewed a number of times prior to baselining
% 	it or sending it for approval. The quality of your test plan speaks
% 	volumes about the quality of the testing you or your team is going to
% 	perform.
%
%	Update the plan as and when necessary. An out-dated and unused
% 	document stinks and is worse than not having the document in the
% 	first place.
%
%%%%%%%%%%%%%%%%%%%%%%%%%%%%%%%%%%%%%%%%%%%%%%%%%%%%%%%%%%%%%%%%%%%%%%%%%%%

\section{Test Items} \hypertarget{labelp}{}
\label{sec:testItems}
The object under test is the Reporting module in version 1.0 of the Research Support System.

\section{Features to be Tested}
\label{sec:featuresToTest}
The features to be tested are divided into functional and non-functional tests. The aspects that will be tested and evaluated in these tests are described in the following sections.
Describe the overall approach for the level of testing. For each major feature or group of features,
specify the approach that will ensure that they are adequately tested. The approach may be described in
sufficient detail to permit identification of the major testing tasks and estimation of the time required to
do each one.

\subsection{Functional}
The approach taken for the functional requirements testing is a black box testing approach. This means that the functionality of the application is tested without considering the internal implementation. Hence, one is aware of what the specific module should be able to accomplish but not how the module accomplishes it. Hence, the result returned by a given function is tested. Test cases are established based on the Software Requirements Specifications and the pre and post conditions for the Reporting module. Both negative and positive tests will be conducted. There are two main services, firstly to generate a report based on the progress status of the publications and secondly to create a report with information regarding all accreditation units active within a specified time period. Tests will be created to test whether the Reporting module performs the right functions to provide these services accurately.
\par The following table displays the two main services that this module consists of and will be tested as well as the ID's of their respective test cases.

\begin{tabular}{ l|L{3cm}|L{3cm}|l }
	\hline
	\textbf{Feature ID} & \textbf{RDS Source} & \textbf{Summary} & \textbf{Test Case ID}\\
	\hline
	\hline
	getAccreditationReport & RDS Document, Section 2.4 bullet 1, page 18 & This service ensures that a user can generate a report with information regarding all accreditation units active within a specified time period  &\begin{tabular}[t]{@{}l@{}}
		\ref{ti_ar_entity}\\
		\ref{ti_ar_pubtype}\\
		\ref{ti_ar_lcs}\\
		\ref{ti_ar_com}\\
	\end{tabular}\\
	\hline
	getProgressReport & RDS Document, Section 2.4 bullet 2, page 19 & This service ensures that a progress report can be generated containing relevant information and filtered information & \begin{tabular}[t]{@{}l@{}}
		\ref{ti_sr_entity}\\
		\ref{ti_sr_pubtype}\\
		\ref{ti_sr_com}\\
	\end{tabular}\\
	\hline
	\hline
\end{tabular}

\subsection{Non-Functional}
\begin{enumerate}
\item For the non-functional testing, the code will firstly be evaluated to see whether there is flexibility in terms of the reporting framework used i.e. that the application functionality is not tied to one specific technology. In addition, the persistence provider should also be easily modifiable.
\item The maintainability of the code will be evaluated in terms it being easy to understand by future developers who would work on the system. In addition, developers should be able to add and modify the functions in the classes with ease. The technologies chosen for the project will also be evaluated to ensure that they will still be available for a reasonable number of years following the deployment of the application.
\item The system should be reliable and provide a deployment without single points of failure. Reliability of this module will be evaluated
\item The testability of the system will be examined by mainly detecting the use of mock objects and automated unit and integration tests. These tests will further be examined as to whether they test whether the service contract in the specification was adhered to.
\item The deployability of the module will be evaluated to see whether the application could be deployed on an application server.
\end{enumerate}

\section{Features not to be Tested}
\label{sec:featuresNotToTest}
Features that are excluded from this test plan include:
\begin{itemize}
	\item Features in modules outside of the Reporting module.
	\item Performance
	\item Scalability
\end{itemize}
For this level of testing, modules were allocated to different teams and therefore this testing activity is restricted to one module. Since the focus is only on verification of requirements, several important non-functional requirements that did not feature in the RDS are also not going to be tested.

\section{Test Cases}
\label{sec:testId}
\hypertarget{label}{}

\subsection{Accreditation Reporting}
Each report generated by this functionality, must be constrained by a time-period. Each test conducted in this section will use the same time period of 2016/02/01 - 2016/05/31.

\subsubsection{By Entity} \label{ti_ar_entity}
\paragraph{Objective:}
The purpose of this test is to evaluate whether or not the Reporting module is capable of
generating a report of accreditation, filtered by Entity. An Entity may be either a
person or group.
\paragraph{Input:}
The following inputs will be used to test this functionality:
\begin{enumerate}
	\item An existing researcher Entity
	\item A non-existing researcher Entity
	\item An existing research group Entity
	\item A non-existing research group Entity
\end{enumerate}
\paragraph{Outcome:}
Each report should contain only information regarding or related to the accreditation target of the Entity. \\
The following are the expected outcomes for a pass result for the functionality:
\begin{enumerate}
	\item A report containing information regarding or related to the accreditation target of the specified researcher.
	\item A report containing information regarding or related to the accreditation target of the research group, where the data is an aggregated sum of the individual researchers in the group.
	\item An empty report.
	\item An empty report.
\end{enumerate}

\subsubsection{By Publication Type} \label{ti_ar_pubtype}
\paragraph{Objective:}
The purpose of this test is to evaluate whether or not the Reporting module is capable of
generating a report of accreditation, filtered by PublicationType.
\paragraph{Input:}
The test will be conducted for each of the existing PublicationTypes, though this may change in time as new PublicationTypes are added to the system. As of writing, these include:
\begin{enumerate}
	\item Journal
	\item Conference
	\item Book Chapter
\end{enumerate}
\paragraph{Outcome:}
Below are the outcomes for each of the specified inputs:
\begin{enumerate}
	\item A report of all publications of PublicationType: Journal, where the accreditation unit value, if published, is included.
	\item A report of all publications of PublicationType: Conference, where the accreditation unit value, if published, is included.
	\item A report of all publications of PublicationType: Book Chapter, where the accreditation unit value, is included.
\end{enumerate}

\subsubsection{By Life-cycle State} \label{ti_ar_lcs}
\paragraph{Objective:}
The purpose of this test is to evaluate whether or not the Reporting module is capable of
generating a report of accreditation, filtered by LifeCylceState.
\paragraph{Input:}
The test will be conducted for each of the existing LifeCycleStates, indicated in the RDS. These include:
\begin{enumerate}
	\item InProgress
	\item Submitted
	\item Accepted
	\item InRevision
	\item Published
	\item Abandoned
\end{enumerate}

\paragraph{Outcome:}
For each of the inputs above, a report of all publications in that state exclusively, where the accreditation unit value is included, must be generated.

\subsubsection{By Some Combination of \ref{ti_ar_entity}, \ref{ti_ar_pubtype} \& \ref{ti_ar_lcs}} \label{ti_ar_com}
\paragraph{Objective:}
This test will evaluate whether or not it is possible to combine multiple filters and generate a report where the result is the concatenation of the applied filters.
\paragraph{Input:}
All possible combinations of \ref{ti_ar_entity}, \ref{ti_ar_pubtype} \& \ref{ti_ar_lcs} will be used in this test.
\paragraph{Outcome:}
Each combination must return a report that only shows information that is within the filters' constraints.

\subsection{Research Status Reporting}
Information generated in this reporting feature should be focused the percentage complete for research publications with a LifeCycleState of: InProgress.

\subsubsection{By Entity} \label{ti_sr_entity}
\paragraph{Objective:}
This test will evaluate whether or not it is possible to filter all InProgress publications by a specific researcher or research group.
\paragraph{Input:}
The following inputs will be used in this test:
\begin{enumerate}
	\item An existing researcher Entity
	\item A non-existing researcher Entity
	\item An existing research group Entity
	\item A non-existing research group Entity
\end{enumerate}
\paragraph{Outcome:}
The following are the expected outcomes for a pass result for the functionality:
\begin{enumerate}
	\item A report containing InProgress publications for the specified researcher.
	\item A report containing InProgress publications for all researchers in the specified research group.
	\item An empty report.
	\item An empty report.
\end{enumerate}

\subsubsection{By Publication Type} \label{ti_sr_pubtype}
\paragraph{Objective:}
This test will evaluate whether or not it is possible to filter all InProgress publications by a specific PublicationType.
\paragraph{Input:}
The publications types to be tested are the same as in TC \ref{ti_ar_pubtype}.
\paragraph{Outcome:}
For each PublicationType as input, only those (InProgress) publications of that type should be returned in the report.

\subsubsection{By Combination of \ref{ti_sr_entity} \& \ref{ti_sr_pubtype}} \label{ti_sr_com}
\paragraph{Objective:}
This test will evaluate whether or not it is possible to combine the filters identified in TC \ref{ti_sr_entity} and TC \ref{ti_sr_pubtype}, and generate a report where the result is the concatenation of the applied filters.
\paragraph{Input:}
Both filters will be specified in the report generation request.
\paragraph{Outcome:}
A report filtered by both Entity and PublicationType should be returned.

\section{Item Pass/Fail Criteria}
\label{sec:passFailCrit}

Each item tested must meet a certain criteria in order to pass. These criteria are as follows:
\begin{itemize}
	\item Each item must adhere to the interface requirements specified in the RDS.
	\item Each item must adhere to their pre- and post-conditions specified in the RDS.
	\item As a result of the previous criterion, each item must produce the correct output for a number of inputs during unit testing.
\end{itemize}
If any of the above criteria are not met, the item will be considered failed.

\section{Test Deliverables}
\label{sec:testDeliverables}
Artefacts to be produced as part of unit testing include the following:
\begin{itemize}
	\item \hyperlink{labelp}{Test Plan}
	\item \hyperlink{labelr}{Test Report}
\end{itemize}
These artefacts have been combined in this master document.

\section{Test Environment}
\label{sec:testEnv}
A sandboxed environment will be set up to provide reliable and deterministic unit testing. Frameworks will be used to allow for testing code to be more succinct and readable since tests have to be approved by non-technical personnel. \\

\textbf{Programming Languages:}\\
The Research Support System is predominantly written in Java, and the Reporting 
module in its current state is written solely in Java. \\

\textbf{Testing Frameworks:}\\
The following testing frameworks and extensions shall used for the Reporting module:
\begin{itemize}
	\item JUnit for functional requirements, 
	\item Mockito for mocking required objects for functional requirements.
\end{itemize}

\textbf{Coding Environment:}\\
The following IDEs are used for test development and Maven build processes:
\begin{itemize}
	\item IntelliJ IDEA
	\item NetBeans
\end{itemize}

\textbf{Operating System:} \\
Tests will be conducted on the following operating systems:
\begin{itemize}
	\item Windows 10
	\item Ubuntu 14 (Linux Distribution)
\end{itemize}
The Reporting module is written in Java, which is cross-platform, so the 
expected test results should remain independent of OS environment.\\

\textbf{Internet Browsers:}\\
The following browsers shall be used in the evaluation of user interface 
components of the Reporting module:
\begin{itemize}
	\item Google Chrome (Version 50.0.2661.102 m)
	\item Mozilla FireFox (Version 46.0.1)
	\item Internet Explorer 11
\end{itemize}

\section{Risks}
\label{risks}
	The following potential risks are found in this module:
\begin{itemize}
	\item Aggregating data for those whose names are incoherent from publication to publication.
	\item Values  which are out of bounds.
	\item Database is in an unusable state.
	\item Too much data to process at once.
\end{itemize}

	The following are contingencies which could be used to prevent the potential risks from arising:
\begin{itemize}
	\item Do a smart lookup on names and research groups.
	\item Always do bounds checking.
	\item Check that the database is in a stable state before attempting to access it.
	\item Use a map-reduce if necessary.
\end{itemize}

\section{Assumptions and Dependencies}
\label{sec:assumptions}
\begin{itemize}
	\item The first assumption is that the Jasper Report generated by the module consists of tables with columns for the name of person/group, publication name, publication type, organization associated with publication type, lifecycle state.
	\item The second assumption is that the module abides by the service contracts and implements the services as specified in the service contracts. In addition, we assume the pre and post conditions for the module hold and hence these will be tested.
	\item The first dependency is that since Java EE was used for implementation, a Java testing framework will be used for the tests.
\end{itemize}
\hypertarget{labelr}{}
