\section{Introduction}
The Reporting module of the Research Support System, for the University of Pretoria, is responsible for the generation and visualisation of reports for the following:
\begin{itemize}
	\item Accreditation of research done by researchers at the University of Pretoria, customizable using filters indicated in the Requirements and Design Document or RDS (see section \ref{sectionReferences}).
	\item Information pertaining to the status of research projects, customizable by filters indicated in the RDS.
\end{itemize}

This document contains a detailed analysis and evaluation of the Reporting module, in terms of its adherence to the Requirements and Design Document, and its correctness in general. 
The document is split into two sections, namely a Unit Test Plan and a Unit Test Report, in the order they are specified. Each section is described in detail below.

\subsection{Unit Test Plan} 
%The functions within the Reporting module in the Research Support System will be tested with unit tests to determine whether the pre and post conditions specified in the service contracts were met by the module. A plan of said tests are provided in this report. The non functional requirements of the system will also be tested/ evaluated and are also described in this report.


\subsubsection{Purpose}
%The purpose of this document is to give an overview of the test plan for the reporting module. This document will outline the approaches used to test the module. Functional as well as non-functional requirements will be tested. This document will give an indication as to how well the reporting module of the system performs. It will also provide a list of functionalities to be tested and their respective results. This test plan will focus on the requirements outlined in the software requirements specification(SRS) instead of the requirements implamented in the reporting module.

The Unit Test Plan (UTP) provides an outline of the manner in which the tests will be conducted for the Reporting module of the Research Support System, identify the features that require testing, and provide other necessary information pertaining to the test environment, conditions and evaluation criteria for feature compliance.
 
\subsubsection{Scope}
\begin{enumerate}
%Needs to be updated according to final structure
	\item Provide an overview of the test plan.
		
The remainder of this document is structured as follows. It will give references to related documents. This document will   list the features to be tested for both functional and non-functional requirements, it will also list features that will not be tested and reasons thereof. The approach to testing will be mentioned, the passing and failing of tests will be given and for all tests the respective suspension and resumption criteria will be stated. References to supporting work on this test plan will be supplied. The test environment will be outlined. Test efforts/costs will be estimated. The training of staff involved to use the system will be mentioned and the responsibilities of the different users will be mentioned. The risks involved in using this system as well as contingency plans will be given. Assumptions and dependencies of this test plan will be mentioned, and the approvers of this test plan will be listed.

	\item Specify the goals/objectives.

The goal of this test plan is to identify possible flaws in the implementation of the reporting module. If flaws are found they will be detailed in this report. The aim would then be for the implementers to either complete unmet requirements or to fix the implementation of identified incorrect code.

	\item Specify any constraints.

The constraints of developing a test plan for this module is the vague documentation of this module given in the SRS. 
\end{enumerate} 

\subsection{Unit Test Report}
%The functions within the Reporting module in the Research Support System were tested with unit tests to determine whether the pre and post conditions specified in the service contracts were met. The results of the tests are provided in this report. The non functional requirements of the system were also tested/ evaluated and these results are also described in this report.

\subsection{Purpose}
%The purpose of this document is to provide a report of the results of the unit tests that tested the pre and post conditions of the functions contained in the reporting module. In addition the results of the non-functional requirements tests are also provided in this report. Any variances of the test items from their specifications are reported on and reasons for these variances are supplied. The actual testing process has also been evaluated. 

\subsection{Scope}
%Specify the contents and organization of this document. Include references to any information captured in automated tools and not contained in this document.
This document provides the results of tests conducted to test the functional and non functional requirements of the Reporting module of the Research Support System. This document is organized as follows: firstly, a brief summary of the test results is provided in the section titled Overview of Test Results. In the section following the brief summary, a detailed description of the test results is provided where all resolved and unresolved anomalies is identified and described. Any variances of the test items from their specifications is reported on. The reasons for each variance is also stated here. The actual testing process is also evaluated in this section. In the section that follows that one, rationale for decisions made is stated. The final section of the document contains the recommendations and conclusions.

\section{References} \label{sectionReferences}
\subsection{Requirements and Design Document (RDS)}
\hypersetup{
	linkcolor = blue       
}   
The master specification for the system in question resides \href{https://clickup.up.ac.za/bbcswebdav/pid-791422-dt-content-rid-8256558_1/xid-8256558_1}{here} and will be referenced throughout this document.

\subsection{Comments} % remove once this section is complete
For each test item (in our case, the Reporting module), supply references to the following documents if they exist: Master Test
Plan, Level Test Plan (The unit test plan is all we have), Level Test Design, Level Test Cases, Level Test Procedures, Level Test Logs,
and Anomaly Reports.

Introduce the following subordinate sections. This section provides an overview of the test results, all
of the detailed test results, rationale for all decisions, and the final conclusions and recommendations.