% \section{Introduction}
% \subsection{Purpose}
% \subsection{Scope}
% 
% \section{Test Environment \& Conditions}
% 
% %This section should contain items such as testing tools and suites, runtime environments, OS, etc.
% 
% \section{Test Procedures \& Results}
% 
% \subsection{Test 1}
% \subsubsection{Description}
% \subsubsection{Pre-Conditions}
% \subsubsection{Post-Conditions}
% \subsubsection{Inputs}
% \subsubsection{Outputs}
% 
% %More tests go here 
% 
% \section{Test Summary}
% %This section provides a summary of the test scope and results. A table should be provided of all the tests conducted with an indication of pass/fail.


%%%%%%%%%%%%%%%%%%%%%%%%%%%%% GUIDELINES %%%%%%%%%%%%%%%%%%%%%%%%%%%%%%%%%
%
%	Make the plan concise. Avoid redundancy and superfluousness. 
% 	If you think you do not need a section that has been mentioned 
% 	in the template, go ahead and delete that section in your 
% 	test plan.
% 	
% 	Be specific. For example, when you specify an operating system 
% 	as a property of a test environment, mention the OS Edition/Version 
% 	as well, not just the OS Name.
%  	
%	Make use of lists and tables wherever possible. Avoid lengthy 
% 	paragraphs.
% 
%	Have the test plan reviewed a number of times prior to baselining 
% 	it or sending it for approval. The quality of your test plan speaks 
% 	volumes about the quality of the testing you or your team is going to 
% 	perform.
% 	
%	Update the plan as and when necessary. An out-dated and unused 
% 	document stinks and is worse than not having the document in the 
% 	first place.
%
%%%%%%%%%%%%%%%%%%%%%%%%%%%%%%%%%%%%%%%%%%%%%%%%%%%%%%%%%%%%%%%%%%%%%%%%%%%
\section{Introduction}

\subsection{Purpose}

\subsection{Scope}
\begin{enumerate}
	\item Provide an overview of the test plan.
	\item Specify the goals/objectives.
	\item Specify any constraints.
\end{enumerate}

\section{References}

\begin{enumerate}
	\item List the related documents, with links to them if available, including the following:

\begin{enumerate}
	\item Project Plan
	\item Configuration Management Plan
\end{enumerate}
\end{enumerate}

\section{Test Items}

\begin{enumerate}
	\item List the test items (software/products) and their versions.
\end{enumerate}

\section{Features to be Tested}

\subsection{Functional}
\begin{enumerate}
	\item List the features of the software/product to be tested.
	\item Provide references to the Requirements and/or Design specifications of the features to be tested
\end{enumerate}
\subsection{Non-functional}     

\section{Features not to be Tested}

\begin{enumerate}
	\item List the features of the software/product which will not be tested.
	\item Specify the reasons these features won’t be tested.
\end{enumerate}

\section{Approach}

\begin{enumerate}
	\item Mention the overall approach to testing.
	\item Specify the testing levels [if it’s a Master Test Plan], the testing types, and the testing methods [Manual/Automated; White Box/Black Box/Gray Box]
\end{enumerate}

\section{Item Pass/Fail Criteria}

\begin{enumerate}
	\item Specify the criteria that will be used to determine whether each test item (software/product) has passed or failed testing.
\end{enumerate}

\section{Suspension Criteria and Resumption Requirements}

\begin{enumerate}
	\item Specify criteria to be used to suspend the testing activity.
	\item Specify testing activities which must be redone when testing is resumed.
\end{enumerate}

\section{Test Deliverables}

\begin{enumerate}
	\item List test deliverables, and links to them if available, including the following:

\begin{enumerate}
	\item Test Plan (this document itself)
	\item Test Cases
	\item Test Scripts
	\item Defect/Enhancement Logs
	\item Test Reports
\end{enumerate}
\end{enumerate}
~
\section{Test Environment}

\begin{enumerate}
	\item Specify the properties of test environment: hardware, software, network etc.
	\item List any testing or related tools.
\end{enumerate}

\section{Estimate}

\begin{enumerate}
	\item Provide a summary of test estimates (cost or effort) and/or provide a link to the detailed estimation.
\end{enumerate}

\section{Schedule}

\begin{enumerate}
	\item Provide a summary of the schedule, specifying key test milestones, and/or provide a link to the detailed schedule.
\end{enumerate}

\section{Staffing and Training Needs}

\begin{enumerate}
	\item Specify staffing needs by role and required skills.
	\item Identify training that is necessary to provide those skills, if not already acquired.
\end{enumerate}

\section{Responsibilities}

\begin{enumerate}
	\item List the responsibilities of each team/role/individual.
\end{enumerate}

\section{Risks}

\begin{enumerate}
	\item List the risks that have been identified.
	\item Specify the mitigation plan and the contingency plan for each risk.
\end{enumerate}

\section{Assumptions and Dependencies}

\begin{enumerate}
	\item List the assumptions that have been made during the preparation of this plan.
	\item List the dependencies.
\end{enumerate}

\section{Approvals}

\begin{enumerate}
	\item Specify the names and roles of all persons who must approve the plan.
	\item Provide space for signatures and dates. (If the document is to be printed.)
\end{enumerate}